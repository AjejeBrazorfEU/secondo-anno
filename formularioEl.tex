\documentclass{article}

% Language setting
% Replace `english' with e.g. `spanish' to change the document language
%\usepackage[italian]{babel}
\usepackage[utf8]{inputenc}

% Set page size and margins
% Replace `letterpaper' with `a4paper' for UK/EU standard size
\usepackage[a4paper,top=2cm,bottom=2cm,left=3cm,right=3cm,marginparwidth=1.75cm]{geometry}

% Useful packages
\usepackage{amsmath}
\usepackage{graphicx}
\usepackage[colorlinks=true, allcolors=blue]{hyperref}
\usepackage{longtable}
%images
\usepackage[rightcaption]{sidecap}
\usepackage{graphicx}
%comment
\usepackage{comment}

%pdf
\usepackage{pdfpages}

\title{Formulario Elettrotecnica T}
\author{Edoardo Carra'}


\begin{document}
\maketitle

\section{Circuiti elettrici}
\noindent \underline{Def circuito}: interconnessione di multipoli.

\noindent \underline{Def nodo}: punto di incontro di 2 o più rami.

\noindent \underline{Def nodo funzionale}: punto di interconnessione di almeno 3 bipoli.

\noindent \underline{Def ramo}: composizione di bipolo più morsetto

\noindent \underline{Def lato}: elemento topologico formato dalla serie di componenti connessi a 2 nodi funzionali.

\noindent \underline{Def maglia}: insieme di rami che formano una linea chiusa.

\noindent \underline{Def potenza}: $p(t)=v(t)*i(t)$

\noindent \underline{Def energia}: $\int_{t_1}^{t_2} p(t) \,dx$

\noindent \underline{Def risolvere un circuito}: calcolare tensioni e correnti (to),per ogni lato della rete.


\medskip
\noindent\fbox{%
    \parbox{\textwidth}{%
    \begin{center}
     \textbf{Teorema di Tellegen}: la somma delle potenze generate all'interno di un circuito è pari alla somma delle potenze 
assorbite. $\sum_{n = 1}^{N_G} P_n  = \sum_{k = 1}^{N_U}  P_k$
    \end{center}
    }%
}

\subsection{Componenti elettrici}
\noindent \underline{Def componenti passivi}: $E(t) >= 0$  $\forall t$, ciò significa che il bilancio energetico complessivo deve essere positivo.
Cioè il componente ha assorbito più di quello che ha erogato.

\noindent \underline{Def componenti attivi}: se $E(t) <= 0 $ significa che il componente ha assorbito meno di quello che ha erogato.

\noindent \underline{Def controllato in tensione}: $i=f(v)$

\noindent \underline{Def controllato in corrente}: $v=g(i)$

\noindent \underline{Def con memoria(dinamico)}: la tensione e la corrente all'istante t dipendono da grandezze al tempo $t_1 < t$

\noindent \underline{Def senza memoria(adinamico)}: la tensione e la corrente all'istante t non dipendono da grandezze al tempo $t_1 < t$

\subsubsection{Resistenza e conduttanza}
\medskip
\noindent\fbox{%
    \parbox{\textwidth}{%
    \begin{center}
     \textbf{V=RI V=$\dfrac{I}{G}$}

     $[\Omega]$ e $[\Omega^{-1}]$
    \end{center}
    }%
}

\begin{figure}[h!]
    \begin{center}
        \includegraphics[width=0.3\linewidth]{img/Triangolo_Stella.png}
    \end{center}
\end{figure}
\noindent Da triangolo a stella:\begin{itemize}
    \item $R_A=\dfrac{R_{AB}R_{AC}}{R_{AB}+R_{BC}+R_{CA}} $
    \item $R_B=\dfrac{R_{BC}R_{AB}}{R_{AB}+R_{BC}+R_{CA}} $
    \item $R_C=\dfrac{R_{CA}R_{BC}}{R_{AB}+R_{BC}+R_{CA}} $
\end{itemize}

\noindent Da stella a triangolo:\begin{itemize}
    \item $R_{AB}=\dfrac{R_{A}R_{B}+R_{B}R_C+R_{C}R_{A}}{R_C} $
    \item $R_{BC}=\dfrac{R_{A}R_{B}+R_{B}R_C+R_{C}R_{A}}{R_A} $
    \item $R_{CA}=\dfrac{R_{A}R_{B}+R_{B}R_C+R_{C}R_{A}}{R_B} $
\end{itemize}

\noindent Nel caso in cui $R_{xy}=R_{\Delta}$ si ha che $R_{\star}=\dfrac{R_{\Delta}}{3}$. Oppure nel caso in cui $R_x$=$R_{\star}$ si ha che 
$R_{\Delta}=3R_{\star}$


\subsubsection{Condensatore}
\medskip
\noindent\fbox{%
    \parbox{\textwidth}{%
    \begin{center}
     \textbf{I=C$\dfrac{d v(t)}{d t}$}

     $[F]$
     \end{center}
    }%
}
\begin{itemize}
    \item[-] $P(t)=v(t)i(t)=v(t)C\dfrac{d v(t)}{d t}$ può assorbire potenza ma anche erogarla.
    \item[-] La tensione definisce lo stato energetico: V è una \textbf{variabile di stato} per i componenti capacitativi. $E(t)=\dfrac{CV_2^2}{2} - \dfrac{CV_1^2}{2}$
    \item[-] Il condensatore può comportarsi come un componente attivo oppure come un componente passivo.
    \item[-] È un componente con memoria. 
    \item[-] I condensatori reali si rappresentano con una resistenza in parallelo. (perché non in serie come i generatori?)
    \item[-] Non sono permesse variazioni istantanee della tensione. 
\end{itemize}


\subsubsection{Induttanza}
\medskip
\noindent\fbox{%
    \parbox{\textwidth}{%
    \begin{center}
     \textbf{V=L$\dfrac{d i(t)}{d t}$}

     $[H]$
     \end{center}
    }%
}
\begin{itemize}
    \item[-] $P(t)=v(t)i(t)=i(t)L\dfrac{d i(t)}{d t}$ può assorbire potenza ma anche erogarla.
    \item[-] La corrente definisce lo stato energetico: V è una \textbf{variabile di stato} per i componenti induttivi. 
    $E(t)=\dfrac{LI_2^2}{2} - \dfrac{LI_1^2}{2}$
    \item[-] Il condensatore può comportarsi come un componente attivo oppure come un componente passivo.
    \item[-] È un componente con memoria. 
    \item[-] Le induttanze reali si rappresentano con un'induttanza in serie.(perché non in parallelo come i generatori?)
    \item[-] Non sono permesse variazioni istantanee della corrente.
\end{itemize}

\subsubsection{Generatori reali di corrente e di tensione}
\begin{figure}[h!]
    \begin{center}
        \includegraphics[width=0.3\linewidth]{img/generatori_reali.png}
    \end{center}
\end{figure}

\medskip
\noindent\fbox{%
    \parbox{\textwidth}{%
    \begin{center}
     \textbf{Teorema del generatore equivalente}: un generatore di corrente A in parallelo con una resistenza R è equivalente
     a un generatore di tensione E in serie alla stessa resistenza R, di valore $E=AR$. (Si dimostra con Thevenin/Norton)
     \end{center}
    }%
}

\begin{figure}[h!]
    \begin{center}
        \includegraphics[width=0.3\linewidth]{img/teorema_del_generatore.png}
    \end{center}
\end{figure}

\noindent Ricorda, due generatori ideali di tensione possono essere messi in serie, ma mai in parallelo. Invece due generatori ideali
 di corrente non possono mai essere messi in serie. Questo perché altrimenti non sarebbero rispettate LKT e LKC.


\subsection{Teorema del massimo trasferimento di potenza}

\subsubsection{Generatori reali di corrente e di tensione}
\begin{figure}[h!]
    \begin{center}
        \includegraphics[width=0.7\linewidth]{img/unione.jpg}
    \end{center}
\end{figure}

\noindent Il massimo valore della potenza trasferita dal generatore alla resistenza di carico si ottiene per $R_c=R_l$ 
per un valore $P_{Lmax}=\dfrac{E^2}{4R}$. Dal grafico si può notare che la potenza tende a 0 per $R_c->\infty$ e per $R_c=0$.
Nel primo caso il circuito tende a un aperto non facendo circolare corrente nel circuito, mentre nel secondo caso la resistenza
 tende se $R_c<R_l$ la potenza dissipata su $R_l$ è maggiore di quella dissipata su $R_c$.

\section{Metodi di analisi}
\subsection{Metodo di Tableau}
\noindent Si individuano N nodi e L lati. Si imposta un sistema di N-1 LKC e L-N+1 LKT indipendenti (cioè le maglie non si devono
 intersecare). Il sistema creato sarà composto da 2L equazioni in 2L incognite(2 per ramo).
\subsection{PSE}
Le variabili di rete(effetti) si possono ottenere come sovrapposizione delle risposte dovute alle singole cause. Si procede attivando un 
generatore alla volta, passivando tutti gli altri.

\noindent N.B. non si può applicare il PSE alle potenze, in quanto non si trattano di grandezze lineari.

\subsection{Teorema di Thevenin}
Se il circuito da semplificare non è accoppiato con il carico è possibile applicare il teorema di Thevenin. 
Qualunque circuito lineare, indipendentemente dalla sua complessità, visto da due terminali è equivalente ad un generatore reale di tensione
costituito da un generatore ideale di tensione in serie a un resistore: l'equivalenza vale limitatamente alla tensione e alla corrente
in corrispondenza dei terminali del circuito.

\noindent $E_{eq}$ è pari alla tensione a vuoto sui morsetti del carico, mentre $Z_{eq}$ è pari all'impedenza equivalente sui terminali del carico.

\subsection{Teorema di Norton}
Se il circuito da semplificare non è accoppiato con il carico è possibile applicare il teorema di Norton. 
Qualunque circuito lineare, comunque complesso, visto da due nodi A-B è equivalente ad un generatore reale di corrente costituito
da un generatore ideale di corrente in parallelo con un resistore: l'equivalenza vale limitatamente alla tensione e alla corrente in 
corrispondenza dei nodi A-B.

\noindent $A_{eq}$ è pari alla corrente di cortocircuito sui morsetti del carico, mentre $Z_{eq}$ è pari all'impedenza equivalente sui
terminali del carico.

\subsection{Teorema di Millman}
Data una rete di 2 o più lati in parallelo, la tensione ai capi della rete è pari al rapporto delle correnti di cortocircuito di ogni singolo lato
(stacco il lato e lo cortocircuito) e la sommatoria delle conduttanze di ogni lato \underline{passivato}.


\medskip
\noindent\fbox{%
    \parbox{\textwidth}{%
    \begin{center}
    $\dfrac{\sum_{k = 1}^{N_{lati}} i_k }{\sum_{j = 1}^{N_{lati}}  \dfrac{1}{R_k} } = \dfrac{\sum_{k = 1}^{N_{tensione}} \dfrac{E_k}{R_k}  + \sum_{y = 1}^{N_{corrente}} I_y }{\sum_{j = 1}^{N_{lati}}  \dfrac{1}{R_k} } $
    \end{center}
    }%
}

\begin{figure}[h!]
    \begin{center}
        \includegraphics[width=0.6\linewidth]{img/millman.jpg}
    \end{center}
\end{figure}
\pagebreak

\section{Transitori}
\noindent Durante il transitorio ci interessiamo delle variazioni delle variabili di stato dei componenti nel tempo. 
Queste sono particolarmente interessanti perché descrivono il comportamento dell'energia del nostro sistema.

\begin{enumerate}
    \item \underline{Condensatore}: $E=\dfrac{1}{2}LI^2$
    \item \underline{Induttore}: $E=\dfrac{1}{2}CV^2$
\end{enumerate}

\medskip
\noindent\fbox{%
    \parbox{\textwidth}{%
    \begin{center}
        \textbf{Postulato di continuità dell'energia:} l'energia varia con continuità, perciò le variabili di stato non possono subire delle 
        discontinuità.
    \end{center}
    }%
}
\medskip

\subsection{Risoluzione dei circuiti}
\noindent Le LKT portano alla risoluzione di equazioni differenziali di ordine N, dove N è il numero di componenti dinamici presenti nel 
nostro sistema. Tutti i circuiti, lineari, trattati in questo corso sono ipoteticamente risolubili con equazioni differenziali ordinarie
a coefficienti costanti(ODE). Queste insieme alle condizioni iniziali del sistema si riducono ad un unico problema di Cauchy.

\noindent N.B. conoscere lo stato iniziale del sistema significa conoscere lo stato energetico, quindi le variabili di stato prima dell'inizio
del transitorio.

\medskip
\noindent\fbox{%
    \parbox{\textwidth}{%
    \begin{center}
        $a_n\dfrac{d^n x(t)}{d t^n}+a_{n-1}\dfrac{d^{n-1} x(t)}{d t^{n-1}} + \dots + a_ox(t)=b(t)$
    \end{center}
    }%
}
\medskip

\noindent Dove $x(t)$ rappresenta la grandezza incognita, $a_i$ rappresentano i coefficienti costanti e $b(t)$ è il termine noto(generatori ecc...).

\noindent La soluzione del nostro sistema sarà quindi formata dalla somma di due contributi:
\begin{enumerate}
    \item La soluzione particolare $p(t)$, che avrà la stessa evoluzione temporale di $b(t)$. (se $b(t)=k$ anche $p(t)=c$)
    \item La soluzione dell'omogenea associata $o(t)$, la quale rappresenta \textbf{l'evoluzione libera del sistema} con i generatori 
    passivati. Nella maggior parte dei casi 
    
    \noindent $o(t)=k_1e^{\lambda_1t}+k_2e^{\lambda_2t}+\dots+k_ne^{\lambda_nt}$ con $\lambda_1,\lambda_2,\dots,\lambda_n < 0$ per la convergenza
    per $t->\infty$. Per ogni esponenziale si calcola una costante di tempo $\tau_n=\dfrac{1}{\lambda_n}[s]$, il quale rappresenta la rapidità
    dell'evoluzione libera del sistema. Più piccolo sarà $\tau$, minore sarà il tempo di transitorio. Il transitorio per convenzione infatti,
    ha durata pari a $T=5max\{\tau_1,\tau_2,\dots,\tau_n\}$. 

    \noindent N.B. i $\tau_i$ non sono associati ai dispositivi dinamici, ma rappresentano un fenomeno transitorio di interazione tra i componenti 
    dinamici. (vedi circuito RLC)
\end{enumerate}

\pagebreak

\subsubsection{Circuito RC}
\medskip
\noindent\fbox{%
    \parbox{\textwidth}{%
    \begin{center}
        $v_C(t)=(V_{C0}-E)e^{-\dfrac{t}{RC}} + E = V_{C0}e^{-\dfrac{t}{RC}} + E(1-e^{-\dfrac{t}{RC}})$
        \medskip

        $\tau=RC$
    \end{center}
    }%
}
\medskip

\begin{figure}[h!]
    \begin{center}
        \includegraphics[width=0.5\linewidth]{img/RC.jpg}
    \end{center}
\end{figure}

\begin{itemize}
    \item[-] $V_{C0}-E)e^{-\dfrac{t}{RC}}$ è detta \underline{risposta transitoria};
    \item[-] $E$ è detta \underline{risposta a regime}.
    \item[-] $V_{C0}e^{-\dfrac{t}{RC}}$ è detta \underline{evoluzione libera}, cioè l'evoluzione senza il generatore. Per $t->\infty$ sarà uguale a 0.
    \item[-] $E(1-e^{-\dfrac{t}{RC}})$ è detta \underline{risposta forzata}, cioè l'evoluzione dettata dal generatore e che si avrebbe
    nel caso di $V_{C0}=0$. Per $t->\infty$ sarà uguale alla \textit{forzante}.
    \item[-] $\dfrac{d v(0)}{d t}=\tau$
    \item[-] $\tau=RC$ perché maggiore è la capacità, maggiore sarà il tempo utilizzato dal condensatore per caricarsi. Inoltre maggiore 
    è la resistenza, minore sarà la corrente e quindi ci impiegherà più tempo per caricarsi.
\end{itemize}


\begin{figure}[h!]
    \begin{center}
        \includegraphics[width=0.5\linewidth]{img/RC_graph.png}
    \end{center}
\end{figure}

\noindent Per quanto riguarda la corrente è possibile notare la presenza di una discontinuità nel punto t=0. Per questo motivo la corrente non è
una variabile di stato per il condensatore. $i(t)=0$ per $t<0$, $i(t)=\dfrac{E-V_{C0}}{R}e^{-\dfrac{t}{RC}}$ per $t>0$.
\pagebreak

\noindent Nel caso reale la presenza di induttanze parassite non permette tale variazione istantanea.
\begin{figure}[h!]
    \begin{center}
        \includegraphics[width=0.7\linewidth]{img/RC_current_graph.png}
    \end{center}
\end{figure}


\subsubsection{Circuito RL}

\medskip
\noindent\fbox{%
    \parbox{\textwidth}{%
    \begin{center}
        $i_L(t)=(I_{L0}-\dfrac{E}{R})e^{-\dfrac{Rt}{L}} + \dfrac{E}{R} = I_{L0}e^{-\dfrac{Rt}{L}}+ \dfrac{E}{R}(1-e^{-\dfrac{Rt}{L}})$
        \medskip

        $\tau=\dfrac{L}{R}$
    \end{center}
    }%
}
\begin{figure}[h!]
    \begin{center}
        \includegraphics[width=0.5\linewidth]{img/RL.png}
    \end{center}
\end{figure}

\medskip
\begin{itemize}
    \item[-] $(I_{L0}-\dfrac{E}{R})e^{-\dfrac{Rt}{L}}$ è detta \underline{risposta a transitorio};
    \item[-] $\dfrac{E}{R}$ è detta \underline{risposta a regime}.
    \item[-] $I_{L0}e^{-\dfrac{Rt}{L}}$ è detta \underline{evoluzione libera}, cioè l'evoluzione senza il generatore. Per $t->\infty$ sarà uguale a 0.
    \item[-] $\dfrac{E}{R}(1-e^{-\dfrac{Rt}{L}})$ è detta \underline{risposta forzata}, cioè l'evoluzione dettata dal generatore e che si avrebbe
    nel caso di $V_{C0}=0$. Per $t->\infty$ sarà uguale alla \textit{forzante}.
    \item[-] $\dfrac{d i(0)}{d t}=\tau$
    \item[-] $I_{L0}$ è la corrente che circola nella maglia di destra prima del tempo zero.
\end{itemize}

\pagebreak

\begin{figure}[h!]
    \begin{center}
        \includegraphics[width=0.5\linewidth]{img/RC_graph.png}
    \end{center}
\end{figure}

\noindent Per quanto riguarda la tensione è possibile notare la presenza di una discontinuità nel punto t=0. Per questo motivo la tensione non è
una variabile di stato per l'induttanza. $v(t)=0$ per $t<0$, $v(t)=(E-RI_{L0})e^{-\dfrac{Rt}{L}}$ per $t>0$.
Nel caso reale la presenza di capacità parassite non permette tale variazione istantanea.
\begin{figure}[h!]
    \begin{center}
        \includegraphics[width=0.7\linewidth]{img/RC_current_graph.png}
    \end{center}
\end{figure}

\noindent N.B. Anche se la tensione non è una variabile di stato per l'induttanza, è di fondamentale importanza per l'analisi dei picchi di corrente
durante il transitorio (alcuni componenti possono essere sottoposti a certo tetto massimo di corrente).

\subsubsection{Confronto tra RL e RC}

\medskip
\noindent\fbox{%
    \parbox{\textwidth}{%
    \begin{center}
        Per i circuiti RC e RL vale: $\frac{dx(t)}{dt} + \dfrac{x(t)}{\tau} = X_{forzante}$
\medskip

        con $x(t)=(x_0-x_\infty)e^{-\dfrac{t}{\tau}}+x_\infty$
    \end{center}
    }%
}
\medskip

\noindent Dove: $x(t)$ è la grandezza incognita, $x_0$ è il valore dell'incognita al tempo $t_0$ e $x_\infty$ è il valore
dell'incognita a regime. In generale, per risolvere un circuito del primo ordine, si può procedere nel seguente modo:

\begin{enumerate}
    \item Ricavare $x(0^-)$:sostituire i condensatori con un aperto e gli induttori con un corto. Per postulato di continuità dell'energia
     $v_{C}(0^-)=v_{C}(0^+)$ e $i_{L}(0^-)=i_{L}(0^+)$.
    \item Ricavare $x(\infty)$: Sostituire i condensatori e gli induttori con un aperto oppure con un corto.
    \item Determinare la resistenza equivalente $R_{eq}$ ”vista” dal condensatore induttore passivando i generatori di tensione e di corrente.
\end{enumerate}

\pagebreak

\subsection{Circuito RLC}
\medskip
\noindent\fbox{%
    \parbox{\textwidth}{%
    \begin{center}
        $i(t)=Ae^{\lambda_1t}+Be^{\lambda_2t}$
        \medskip

        $\alpha=\dfrac{R}{2L}$ coefficiente di smorzamento. Determina la velocità con cui la corrente tende a zero.
        \medskip

        $\omega_0^2=\dfrac{1}{LC}$ pulsazione di risonanza.
        \medskip

        $\Delta=\alpha^2-\omega_0^2$ 
    \end{center}
    }%
}

\begin{figure}[h!]
    \begin{center}
        \includegraphics[width=0.5\linewidth]{img/RLC.png}
    \end{center}
\end{figure}

\subsubsection{$\Delta>0$ risposta sovrasmorzata}

\medskip
\noindent\fbox{%
    \parbox{\textwidth}{%
    \begin{center}
        $i(t)=\dfrac{E-V_{C0}}{2L\sqrt{\Delta}}(e^{\sqrt{\Delta}t}-e^{-\sqrt{\Delta}t})e^{-\alpha t}$
    \end{center}
    }%
}

\begin{figure}[h!]
    \begin{center}
        \includegraphics[width=0.8\linewidth]{img/sovrasmorzata.png}
    \end{center}
\end{figure}


\subsubsection{$\Delta=0$ smorzamento critico}
\medskip

\noindent\fbox{%
    \parbox{\textwidth}{%
    \begin{center}
        $i(t)=\dfrac{E-V_{C0}}{L}te^{-\alpha t}$
    \end{center}
    }%
}

\begin{figure}[h!]
    \begin{center}
        \includegraphics[width=0.8\linewidth]{img/smorzamento_critico.png}
    \end{center}
\end{figure}

\subsubsection{$\Delta<0$ risposta sovrasmorzata}
\medskip

\noindent\fbox{%
    \parbox{\textwidth}{%
    \begin{center}
        $i(t)=\dfrac{E-V_{C0}}{L\omega_d}\sin(\omega_dt)e^{-\alpha t}$
        \medskip

        $\omega_d=\omega_0^2-\alpha^2$
    \end{center}
    }%
}

\noindent in questo caso avviene uno scambio di energia tra il condensatore e l'induttore.

\begin{figure}[h!]
    \begin{center}
        \includegraphics[width=0.8\linewidth]{img/sottosmorzata.png}
    \end{center}
\end{figure}

\subsubsection{confronto tra le risposte}
\noindent Quando si studia il transiorio di una rete, è bene tenere conto dei picchi di corrente e della durata del transitorio:
\begin{itemize}
    \item[-] La risposta \textbf{sovrasmorzata} presenta un minor picco di corrente e dinamiche di transitorio più lente. Si entra a regime più tardi.
    \item[-] La risposta \textbf{sottosmorzata} presenta un picco di corrente più alto e dinamiche di transitorio più veloci. Si entra a regime prima.
\end{itemize}

\noindent Si noti che la tensione su ogni componente per $t->\infty$ sarà pari a $E$, poichè $C$ si comporta come un aperto.

\begin{figure}[h!]
    \begin{center}
        \includegraphics[width=0.8\linewidth]{img/confronto.png}
    \end{center}
\end{figure}

\subsection{Metodo per ispezione}
\begin{enumerate}
    \item Se $v_{Ci}(0^-)$ e $i_{Lk}(0^-)$ non `e nota, ricavarla dal circuito in $t = 0^{-}$ sostituendo
    i condensatori con un aperto e gli induttori con un corto cirucito. Per postulato di continuità dell'energia $v_{Ci}(0^-)=v_{Ci}(0^+)$ e
     $i_{Lk}(0^-)=i_{Lk}(0^+)$.
    \item Ricavare $x(0)$: nel circuito all'istante $t = 0^+$, sostituire i condensatori e gli induttori con
    un generatore indipendente di tensione o di corrente di valore $v_Ci(0)$ oppure di $i_Lk(0)$.
    \item Ricavare $x(\infty)$: Sostituire i condensatori e gli induttori con un aperti oppure  corti.
\end{enumerate}

\section{Regime sinusoidale}
\noindent \underline{Def periodica}: una grandezza tempo invariante $a(t)$ si dice periodica se vale $a(t)=a(t+T)$.

\noindent \underline{Def valor medio}: $A_m=\dfrac{1}{T}\int_{0}^{T} a(t)dx $.

\noindent \underline{Def valor efficace}: $A_{eff}=\sqrt{\dfrac{1}{T}\int_{0}^{T} a^2(t)dx }$.
\medskip

\noindent In questo corso ci interessiamo dei segnali periodici alternati con $A_m=0$, della forma: 
\medskip

\noindent $a(t)=$\^{A}$\cos(\omega t+\alpha)$. Dove \^{A} è il valore di picco della sinusoide, $\omega$ è la pulsazione e
$\alpha$ è la fase.

\noindent \underline{Def $A_{eff,cos}=A$}=$\dfrac{\hat{A}}{\sqrt{2}}$. Il $\sqrt{2}$ è detto fattore di forma.

\noindent La potenza media dissipata da un resistenza è pari a $<P(t)>=\dfrac{R}{T}\int_{0}^{T} i(t)^2dx=\dfrac{R\hat{I}}{2}$. Si noti 
quindi che una corrente continua di 1 A, genera mediamente le stesse perdite di un corrente alternata di valore efficace pari a 1 A.
\medskip

\noindent Dati $a(t)=$\^{A}$\cos(\omega t+\alpha)$ e $b(t)=$\^{A}$\cos(\omega t+\beta)$ si definiscono:
\medskip

\noindent \underline{Def sfasamento}: $\phi = \alpha - \beta$. Si noti che lo sfasamento è indipendente dal sistema di riferimento temporale.

\noindent \underline{Def} $a(t)$ si dice in anticipo rispetto a $b(t)$ se $\phi>0$.

\noindent \underline{Def} $a(t)$ si dice in ritardo rispetto a $b(t)$ se $\phi<0$

\noindent \underline{Def} $a(t)$ si dice in fase con $b(t)$ se $\phi=0$

\noindent \underline{Def} $a(t)$ si dice in quadratura in anticipo/ritardo con $b(t)$ se $\phi=\pm \dfrac{\pi}{2}$

\subsection{Rappresentazione fasoriale}
\noindent Una sinusoide può essere vista come la parte reale di un vettore nel piano complesso rotante:
$a(t)=\Re(\hat{A}e^{j(\omega t + \phi)})=\Re(A\sqrt{2}e^{j(\omega t)}e^{j\phi})$. Il vettore rotante può quindi essere scomposto in 2 parti:
\textit{vettore rotante} $\sqrt{2}e^{j(\omega t)}$ e il \textit{fasore} $Ae^{j\phi}$. È quindi possibile rappresentare un sinusoide utilizzando 
il fasore corrispondente, il quale contiene le informazioni energetiche e di fase.

\subsection{Trasformata di Steinmetz}
\noindent Si definisce trasformata di Steinmetz:

\medskip
\noindent\fbox{%
    \parbox{\textwidth}{%
    \begin{center}
        $S[a(t)]=\dfrac{\sqrt{2}}{T}\int_{0}^{T} a(t)e^{-j \omega t}dx $
    \end{center}
    }%
}

\noindent Da cui è possibile ricavare la trasformata di Steinmetz di una sinusoide: $S[a(t)]=Ae^{j\alpha}$ che è pari 
al fasore rappresentativo.

\subsubsection{Proprietà}
\begin{enumerate}
    \item \underline{Operatore lineare:} $S[n\cdot a(t)+m\cdot b(t)]=n\cdot S[a(t)]+m\cdot S[b(t)]$.
    \item \underline{Trasformata della derivata:} $S[\dfrac{d a(t)}{dt}]=j\omega S[a(t)]$. La derivata è sfasata di 90 gradi e 
    amplificata di un fattore pari a $\omega$.
\end{enumerate}

\subsection{Metodo simbolico}
\noindent Utilizziamo la trasformata di Steinmetz per risolvere i circuiti in regime sinusoidale. Come variano i teoremi da 
regime continuo a sinusoidale? $\underline{V}$ indica il fasore ed è uguale a $\underline{V}=Ve^{j\phi}$.

\noindent \underline{LKT}: $\sum \underline{V}_i=0$ 

\noindent \underline{LKC}: $\sum \underline{I}_k=0$ 

\subsection{Equazioni costitutive in forma simbolica}
\noindent Si definisce \textbf{impedenza}:

\medskip
\noindent\fbox{%
    \parbox{\textwidth}{%
    \begin{center}
        $\underline{Z}=\dfrac{Ve^{j\alpha}}{Ie^{j\beta}}=\dfrac{V}{I}e^{j(\alpha-\beta)}=Ze^{j\phi}$
    \end{center}
    }%
}
\medskip

\noindent L'impedenza rappresenta quindi lo sfasamento e il rapporto tra l'ampiezza della corrente e della tensione introdotto da un
 componente. È nell'impedenza e nella \textbf{legge di Ohm generalizzata} $\underline{V}=\underline{Z}\cdot\underline{I}$
che vanno cercate le equazioni costitutive dei componenti.
\medskip 

\noindent L'induttanza può essere scritta anche nella forma \textbf{R+jX} dove R è la parte reale resistiva e 
la grandezza \textbf{X} è detta \textbf{reattanza} del ramo, e costituisce la parte immaginaria dell'impedenza. La
reattanza dipende dalla capacità e dall'induttanza del ramo, e dalla pulsazione $\omega$ (risposta in frequenza),
 mentre la parte reale dipende solamente dai componenti resistivi.
La reattanza viene distinta in reattanza induttiva $X_L=\omega L$ e capacitiva $X_C=\dfrac{-j}{\omega C}$.


\noindent La fase:\begin{itemize}
    \item[-] È positiva se la tensione è in anticipo rispetto alla corrente (reattanza induttiva).
    \item[-] È negativa se la tensione è in ritardo rispetto alla corrente (reattanza capacitiva).
\end{itemize}
\medskip

\noindent \underline{Generatore sinusoidale di tensione}: $e(t)=\sqrt{2}E cos(\omega t+ \alpha)$ \space $\underrightarrow{S[e(t)]}$ 
\space $\underline{E}e^{j\alpha}$

\noindent \underline{Generatore sinusoidale di corrente}: $i(t)=\sqrt{2}I cos(\omega t+ \beta)$ \space $\underrightarrow{S[i(t)]}$ 
\space $\underline{I} e^{j\beta}$
\pagebreak

\includepdf[pages=70]{elettrotecnica-G-ribani.pdf}

\section{Risonanza}
\medskip

\noindent\fbox{%
    \parbox{\textwidth}{%
    \begin{center}
        Pulsazione di risonanza: $\omega_0=\dfrac{1}{\sqrt{CL}}$
    \end{center}
    }%
}
\medskip

\noindent Si cerca un valore di $\omega$ tale per cui la reattanza del circuito risulta essere zero: $\omega L - \dfrac{1}{\omega C}=0$.
Con l'annullarsi a vicenda di tali componenti, l'impedenza del circuito, alla frequenza alla quale si verifica la risonanza, sarà data 
dal solo contributo dei componenti resistivi. In particolare essa avrà modulo minimo e fase nulla. 


\begin{figure}[h!]
    \begin{center}
        \includegraphics[width=0.8\linewidth]{img/risonanza_serie.png}
    \end{center}
\end{figure}

\noindent Come si può notare dalla formula di ampiezza e di sfasamento della corrente, per $\omega\rightarrow0$ vince il comportamento
della capacità che ha reattanza massima e la tensione è in quadratura in ritardo rispetto alla corrente, mentre per $\omega\rightarrow\infty$
 vince l'induttanza della capacità che ha reattanza massima e la tensione è in quadratura in anticipo rispetto alla corrente.
 Se $\omega<\omega_0$ il circuito è \textbf{ohmico-capacitivo} e $X_C>X_L$, mentre se $\omega>\omega_0$  il circuito è \textbf{ohmico-induttivo}
 $X_C<X_L$.


\begin{figure}[h!]
    \begin{center}
        \includegraphics[width=0.9\linewidth]{img/risonanza_grafici.png}
    \end{center}
\end{figure}

\begin{figure}[h!]
    \begin{center}
        \includegraphics[width=0.9\linewidth]{img/risonanza_vettori.png}
    \end{center}
\end{figure}

\subsection{Antirisonanza}
\medskip

\noindent\fbox{%
    \parbox{\textwidth}{%
    \begin{center}
        Pulsazione di antirisonanza: $\omega_0=\dfrac{1}{\sqrt{CL}}$
    \end{center}
    }%
}
\medskip

\noindent Si cerca un valore di $\omega$ tale per cui la reattanza del circuito risulta essere infinita.
Con l'annullarsi a vicenda di tali componenti, l'impedenza del circuito, alla frequenza alla quale si verifica la risonanza, non permetterà
il passaggio di corrente. In particolare essa avrà modulo $\infty$ e fase massima. 


\begin{figure}[h!]
    \begin{center}
        \includegraphics[width=0.35\linewidth]{img/antirisonanza.png}
    \end{center}
\end{figure}

\noindent Come si può notare dalla formula di ampiezza e di sfasamento della corrente, per $\omega\rightarrow0$ l'induttore si comporta
come un corto annullando la reattanza, mentre per $\omega\rightarrow\infty$ è la capacità che si comporta
come un corto annullando la reattanza. Per $\omega\rightarrow\omega_0$ la corrente sul generatore tende a zero.
Questo perché si instaura  un regime periodico di scambio energetico tra il condensatore e l'induttanza. In assenza di dispersioni e di
 resistenze, la circolazione nella maglia costituita dall'induttanza e dal condensatore continua indefinitamente.
\medskip

 \noindent $I_C=\sqrt{\dfrac{C}{L}}Ej$ \space $I_L=-\sqrt{\dfrac{C}{L}}Ej$

\begin{figure}[h!]
    \begin{center}
        \includegraphics[width=0.8\linewidth]{img/antirisonanza_graph.png}
    \end{center}
\end{figure}

\noindent Per $\omega < \omega_0$ la reattanza è positiva, ed il circuito ha un comportamento prevalentemente ohmico-induttivo con
 uno sfasamento positivo. Per basse frequenze la corrente fluisce prevalentemente nel ramo induttivo, che quindi caratterizza 
 maggiormente il comportamento del circuito. Per $\omega > \omega_0$ la reattanza è negativa, ed il circuito ha
prevalentemente una caratteristica ohmico - capacitiva, con sfasamento negativo. Per alte frequenze la
corrente fluisce maggiormente per il ramo capacitivo.
\pagebreak
 
\section{Potenza in regime sinusoidale}
\noindent Un dispositivo sottoposto a una tensione $v(t)=\hat{V}cos(\omega t)$ e a una corrente $v(t)=\hat{I}cos(\omega t-\phi)$
assorbirà/erogherà una potenza pari a $p(t)=v(t)i(t)$. La corrente $i(t)$ può essere scomposta nelle due componenti $i_a$ e $i_r$ , 
dette rispettivamente \textbf{corrente attiva e reattiva}. La corrente attiva è quindi la componente della corrente in fase con la 
tensione, mentre la corrente reattiva è la componente in quadratura. Nota che la corrente reattiva è sfasata sempre di 
$\pm \dfrac{\pi}{2}$. Questo è dovuto unicamente al contributo delle reattanze, le quali si scambiano corrente sempre sfasata di 
$\pm \dfrac{\pi}{2}$. Quanto questa corrente andrà ad incidere sullo sfasamento totale? Dipende unicamente dal rapporto delle parti 
resistive e reattive. 

\noindent \underline{Def corrente attiva}: $i_a(t)=\hat{I}cos(\phi)cos(\omega t)$.

\noindent \underline{Def corrente reattiva}: $i_r(t)=\hat{I}sin(\phi)sin(\omega t)$.

\begin{figure}[h!]
    \begin{center}
        \includegraphics[width=0.6\linewidth]{img/power.png}
    \end{center}
\end{figure}

\noindent La potenza quindi può essere scomposta in due componenti:

\medskip
\noindent\fbox{%
    \parbox{\textwidth}{%
    \begin{center}
        \textbf{Potenza attiva istantanea}: $p_a(t)=\hat{I}\hat{V}cos(\phi)cos^2(\omega t)=2VIcos(\phi)cos^2(\omega t)$.

        \textbf{Potenza reattiva istantanea}: $p_r(t)=\dfrac{\hat{I}\hat{V}}{2}sin(\phi)sin(2\omega t)=VIsin(\phi)sin(2\omega t)$.
    \end{center}
    }%
}
\medskip

\begin{figure}[h!]
    \begin{center}
        \includegraphics[width=0.6\linewidth]{img/power_ar.png}
    \end{center}
\end{figure}

\noindent Dal grafico è possibile evincere una serie di considerazioni:
\begin{itemize}
    \item[-] La potenza attiva è sempre maggiore di zero. Questo significa che si tratta di una potenza che è sempre fornita dal generatore 
    all'utilizzatore. La potenza attiva è detta potenza utile.
    \item[-] La potenza reattiva invece ha frequenza doppia rispetto alle altre grandezze. Essa rappresenta lo scambio di energia conservativo 
    che avviene tra il generatore e i componenti dinamici del circuito. La potenza reattiva è quindi un indicatore
    di uno scambio di energia di tipo conservativo.
\end{itemize}

\noindent Ci interessiamo a capire quali sono il valore medio della potenza istantanea ed energia fornita al carico calcolate sul periodo T:

\medskip
\noindent\fbox{%
    \parbox{\textwidth}{%
    \begin{center}
        \textbf{Energia assorbita dal carico in un periodo T}: $<E>=\int_{0}^{T} p(t)dx=\int_{0}^{T} p_a(t)dx+\int_{0}^{T} p_r(t)dx=\int_{0}^{T} p_a(t)dx=VIcos(\phi)T $.
        \medskip

        \textbf{Potenza media assorbita dal carico in un periodo T}: 
        
        $P_a=\dfrac{1}{T}\int_{0}^{T} p(t)dx=VIcos(\phi)$.
    \end{center}
    }%
}
\medskip

\noindent È possibile notare che la potenza, e quindi l'energia, fornita al carico, \underline{non} dipende dalla potenza reattiva.
Questo perché il \textit{bilancio energetico} della potenza reattiva sul periodo T è nullo: reattanze e generatore si scambiano energia in modo 
conservativo. La media della potenza attiva istantanea coincide quindi, con la media della potenza istantanea.
\medskip

\noindent Il $cos(\phi)$ è detto \textbf{fattore di potenza} e gioca un ruolo molto importante: più corrente e tensione saranno sfasati tra loro,
minore sarà la potenza utilizzabile dal carico.

\subsection{Potenza complessa}

\medskip
\noindent\fbox{%
    \parbox{\textwidth}{%
    \begin{center}
        \textbf{Potenza complessa}: $\underline{N}=\underline{V}\cdot\underline{I}^*=Ve^{j\alpha_v}Ie^{-j\alpha_i}=VIe^{j\phi}=VIcos(\phi)+jVIsin(\phi)$
        \medskip

        \textbf{Potenza apparente}: $|\underline{N}|=\sqrt{P^2+Q^2}$
    \end{center}
    }%
}
\medskip

\noindent La potenza complessa permette di rappresentare tutte le parti che compongono la potenza coerentemente alla rappresentazione 
fasoriale. La parte reale rappresenta la potenza attiva \textbf{P}, mentre la parte immaginaria rappresenta la potenza reattiva \textbf{Q},
cioè il massimo della potenza reattiva istantanea. (ATTENZIONE Q non è la media della potenza reattiva, altrimenti sarebbe 0)
\medskip

\noindent Dalla legge di Ohm generalizzata è possibile riscrivere la potenza complessa di un ramo della rete:
$\underline{N}=\underline{Z}\cdot\underline{I}\cdot\underline{I}^*=\underline{Z}\cdot I^2=RI^2+jXI^2$. Quindi $P=RI^2$ e $Q=XI^2$.
La potenza reattiva può quindi essere positiva nel caso di reattanza induttiva, oppure negativa nel caso di reattanza capacitiva.

\medskip
\noindent\fbox{%
    \parbox{\textwidth}{%
    \begin{center}
        \textbf{Unità di misura della potenza}
        \medskip
        \begin{itemize}
            \item[-] P watt [W]
            \item[-] Q volt-ampere reattivi [VAR]
            \item[-] $|\underline{N}|$ volt-ampere [VA]   
        \end{itemize}


    \end{center}
    }%
}
\medskip


\end{document}
