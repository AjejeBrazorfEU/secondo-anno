\documentclass{article}

% Language setting
% Replace `english' with e.g. `spanish' to change the document language
%\usepackage[italian]{babel}
\usepackage[utf8]{inputenc}

% Set page size and margins
% Replace `letterpaper' with `a4paper' for UK/EU standard size
\usepackage[a4paper,top=2cm,bottom=2cm,left=3cm,right=3cm,marginparwidth=1.75cm]{geometry}

% Useful packages
\usepackage{amsmath}
\usepackage{graphicx}
\usepackage[colorlinks=true, allcolors=blue]{hyperref}
\usepackage{longtable}
%images
\usepackage[rightcaption]{sidecap}
\usepackage{graphicx}
%comment
\usepackage{comment}

\title{Formulario Elettrotecnica T}
\author{Edoardo Carra'}


\begin{document}
\maketitle

\section{Circuiti elettrici}
\noindent \underline{Def circuito}: interconnessione di multipoli.

\noindent \underline{Def nodo}: punto di incontro di 2 o più rami.

\noindent \underline{Def nodo funzionale}: punto di interconnessione di almeno 3 bipoli.

\noindent \underline{Def ramo}: composizione di bipolo più morsetto

\noindent \underline{Def lato}: elemento topologico formato dalla serie di componenti connessi a 2 nodi funzionali.

\noindent \underline{Def maglia}: insieme di rami che formano una linea chiusa.

\noindent \underline{Def potenza}: $p(t)=v(t)*i(t)$

\noindent \underline{Def energia}: $\int_{t_1}^{t_2} p(t) \,dx$

\noindent \underline{Def risolvere un circuito}: calcolare tensioni e correnti (to),per ogni lato della rete.


\medskip
\noindent\fbox{%
    \parbox{\textwidth}{%
    \begin{center}
     \textbf{Teorema di Tellegen}: la somma delle potenze generate all'interno di un circuito è pari alla somma delle potenze 
assorbite. $\sum_{n = 1}^{N_G} P_n  = \sum_{k = 1}^{N_U}  P_k$
    \end{center}
    }%
}

\subsection{Componenti elettrici}
\noindent \underline{Def componenti passivi}: $E(t) >= 0 \forall t$

\noindent \underline{Def componenti attivi}: $E(t) <= 0 $ per almeno una t

\noindent \underline{Def controllato in tensione}: $i=f(v)$

\noindent \underline{Def controllato in corrente}: $v=g(i)$

\noindent \underline{Def con memoria}: la tensione e la corrente all'istante t dipendono da grandezze al tempo $t_1 < t$

\noindent \underline{Def senza memoria}: la tensione e la corrente all'istante t non dipendono da grandezze al tempo $t_1 < t$

\subsubsection{Resistenza e conduttanza}
\medskip
\noindent\fbox{%
    \parbox{\textwidth}{%
    \begin{center}
     \textbf{V=RI V=$\dfrac{I}{G}$}

     $[\Omega]$ e $[\Omega^{-1}]$
    \end{center}
    }%
}

\begin{figure}[h!]
    \begin{center}
        \includegraphics[width=0.3\linewidth]{img/Triangolo_Stella.png}
    \end{center}
\end{figure}
\noindent Da stella a triangolo:\begin{itemize}
    \item $R_A=\dfrac{R_{AB}R_{AC}}{R_{AB}+R_{BC}+R_{CA}} $
    \item $R_B=\dfrac{R_{BC}R_{AB}}{R_{AB}+R_{BC}+R_{CA}} $
    \item $R_C=\dfrac{R_{CA}R_{BC}}{R_{AB}+R_{BC}+R_{CA}} $
\end{itemize}

\noindent Da triangolo a stella:\begin{itemize}
    \item $R_{AB}=\dfrac{R_{A}R_{B}+R_{B}R_C+R_{C}R_{A}}{R_C} $
    \item $R_{BC}=\dfrac{R_{A}R_{B}+R_{B}R_C+R_{C}R_{A}}{R_A} $
    \item $R_{CA}=\dfrac{R_{A}R_{B}+R_{B}R_C+R_{C}R_{A}}{R_B} $
\end{itemize}

\noindent Nel caso in cui $R_{xy}=R_{\Delta}$ si ha che $R_{\star}=\dfrac{R_{\Delta}}{3}$. Oppure nel caso in cui $R_x$=$R_{\star}$ si ha che 
$R_{\Delta}=3R_{\star}$


\subsubsection{Condensatore}
\medskip
\noindent\fbox{%
    \parbox{\textwidth}{%
    \begin{center}
     \textbf{I=C$\dfrac{d v(t)}{d t}$}

     $[F]$
     \end{center}
    }%
}
\begin{itemize}
    \item[-] $P(t)=v(t)i(t)=v(t)C\dfrac{d v(t)}{d t}$ può assorbire potenza ma anche erogarla.
    \item[-] La tensione definisce lo stato energetico: V è una \textbf{variabile di stato} per i componenti capacitativi. $E(t)=\dfrac{CV_2^2}{2} - \dfrac{CV_1^2}{2}$
    \item[-] Il condensatore può comportarsi come un componente attivo oppure come un componente passivo.
    \item[-] È un componente con memoria. 
    \item[-] I condensatori reali si rappresentano con una resistenza in parallelo. (perché non in serie come i generatori?)
    \item[-] Non sono permesse variazioni istantanee della tensione. 
\end{itemize}


\subsubsection{Induttanza}
\medskip
\noindent\fbox{%
    \parbox{\textwidth}{%
    \begin{center}
     \textbf{V=L$\dfrac{d i(t)}{d t}$}

     $[H]$
     \end{center}
    }%
}
\begin{itemize}
    \item[-] $P(t)=v(t)i(t)=i(t)L\dfrac{d i(t)}{d t}$ può assorbire potenza ma anche erogarla.
    \item[-] La corrente definisce lo stato energetico: V è una \textbf{variabile di stato} per i componenti induttivi. 
    $E(t)=\dfrac{LI_2^2}{2} - \dfrac{LI_1^2}{2}$
    \item[-] Il condensatore può comportarsi come un componente attivo oppure come un componente passivo.
    \item[-] È un componente con memoria. 
    \item[-] Le induttanze reali si rappresentano con un'induttanza in serie.(perché non in parallelo come i generatori?)
    \item[-] Non sono permesse variazioni istantanee della corrente.
\end{itemize}
\pagebreak

\subsubsection{Generatori reali di corrente e di tensione}
\begin{figure}[h!]
    \begin{center}
        \includegraphics[width=0.3\linewidth]{img/generatori_reali.png}
    \end{center}
\end{figure}

\medskip
\noindent\fbox{%
    \parbox{\textwidth}{%
    \begin{center}
     \textbf{Teorema del generatore equivalente}: un generatore di corrente A in parallelo con una resistenza R è equivalente
     a un generatore di tensione E in serie alla stessa resistenza R, di valore $E=AR$.
     \end{center}
    }%
}

\begin{figure}[h!]
    \begin{center}
        \includegraphics[width=0.3\linewidth]{img/teorema_del_generatore.png}
    \end{center}
\end{figure}

\noindent Ricorda, due generatori ideali di tensione possono essere messi in serie, ma mai in parallelo. Invece due generatori ideali
 di corrente non possono mai essere messi in serie. Questo perché altrimenti non sarebbero rispettate LKT e LKC.


\subsection{Teorema del massimo trasferimento di potenza}

\subsubsection{Generatori reali di corrente e di tensione}
\begin{figure}[h!]
    \begin{center}
        \includegraphics[width=0.7\linewidth]{img/unione.jpg}
    \end{center}
\end{figure}

\noindent Il massimo valore della potenza trasferita dal generatore alla resistenza di carico si ottiene per $R_c=R_l$ 
per un valore $P_{Lmax}=\dfrac{E^2}{4R}$. Dal grafico si può notare che la potenza tende a 0 per $R_c->\infty$ e per $R_c=0$.
Nel primo caso il circuito tende a un aperto non facendo circolare corrente nel circuito, mentre nel secondo caso la resistenza
 tende se $R_c<R_l$ la potenza dissipata su $R_l$ è maggiore di quella dissipata su $R_c$.

\section{Metodi di analisi}
\subsection{Metodo di Tableau}
\noindent Si individuano N nodi e L lati. Si imposta un sistema di N-1 LKC e L-N+1 LKT indipendenti (cioè le maglie non si devono
 intersecare). Il sistema creato sarà composto da 2L equazioni in 2L incognite(2 per ramo).
\subsection{PSE}
Le variabili di rete(effetti) si possono ottenere come sovrapposizione delle risposte dovute alle singole cause. Si procede attivando un 
generatore alla volta, passivando tutti gli altri.

\noindent N.B. non si può applicare il PSE alle potenze, in quanto non si trattano di grandezze lineari.

\subsection{Teorema di Thevenin}
Se il circuito da semplificare non è accoppiato con il carico è possibile applicare il teorema di Thevenin. 
Qualunque circuito lineare, indipendentemente dalla sua complessità, visto da due terminali è equivalente ad un generatore reale di tensione
costituito da un generatore ideale di tensione in serie a un resistore: l'equivalenza vale limitatamente alla tensione e alla corrente
in corrispondenza dei terminali del circuito.

\noindent $E_{eq}$ è pari alla tensione a vuoto sui morsetti del carico, mentre $Z_{eq}$ è pari all'impedenza equivalente sui terminali del carico.

\subsection{Teorema di Norton}
Se il circuito da semplificare non è accoppiato con il carico è possibile applicare il teorema di Norton. 
Qualunque circuito lineare, comunque complesso, visto da due nodi A-B è equivalente ad un generatore reale di corrente costituito
da un generatore ideale di corrente in parallelo con un resistore: l'equivalenza vale limitatamente alla tensione e alla corrente in 
corrispondenza dei nodi A-B.

\noindent $A_{eq}$ è pari alla corrente di cortocircuito sui morsetti del carico, mentre $Z_{eq}$ è pari all'impedenza equivalente sui
terminali del carico.

\subsection{Teorema di Millman}
Data una rete di 2 o più lati in parallelo, la tensione ai capi della rete è pari al rapporto delle correnti di cortocircuito di ogni singolo lato
e la sommatoria delle conduttanze di ogni lato.


\medskip
\noindent\fbox{%
    \parbox{\textwidth}{%
    \begin{center}
    $\dfrac{\sum_{k = 1}^{N_{lati}} i_k }{\sum_{j = 1}^{N_{lati}}  \dfrac{1}{R_k} } = \dfrac{\sum_{k = 1}^{N_{tensione}} \dfrac{E_k}{R_k}  + \sum_{y = 1}^{N_{corrente}} I_y }{\sum_{j = 1}^{N_{lati}}  \dfrac{1}{R_k} } $
    \end{center}
    }%
}

\begin{figure}[h!]
    \begin{center}
        \includegraphics[width=0.6\linewidth]{img/millman.jpg}
    \end{center}
\end{figure}
\pagebreak

\section{Transitori}
\noindent Durante il transitorio ci interessiamo delle variazioni delle variabili di stato dei componenti nel tempo. 
Queste sono particolarmente interessanti perché descrivono il comportamento dell'energia del nostro sistema.

\begin{enumerate}
    \item \underline{Condensatore}: $E=\dfrac{1}{2}LI^2$
    \item \underline{Induttore}: $E=\dfrac{1}{2}CV^2$
\end{enumerate}

\medskip
\noindent\fbox{%
    \parbox{\textwidth}{%
    \begin{center}
        \textbf{Postulato di continuità dell'energia:} l'energia varia con continuità, perciò le variabili di stato non possono subire delle 
        discontinuità.
    \end{center}
    }%
}
\medskip

\subsection{Risoluzione dei circuiti}
\noindent Le LKT portano alla risoluzione di equazioni differenziali di ordine N, dove N è il numero di componenti dinamici presenti nel 
nostro sistema. Tutti i circuiti, lineari, trattati in questo corso sono ipoteticamente risolubili con equazioni differenziali ordinarie
a coefficienti costanti(ODE). Queste insieme alle condizioni iniziali del sistema si riducono ad un unico problema di Cauchy.

\noindent N.B. conoscere lo stato iniziale del sistema significa conoscere lo stato energetico, quindi le variabili di stato prima dell'inizio
del transitorio.

\medskip
\noindent\fbox{%
    \parbox{\textwidth}{%
    \begin{center}
        $a_n\dfrac{d^n x(t)}{d t^n}+a_{n-1}\dfrac{d^{n-1} x(t)}{d t^{n-1}} + \dots + a_ox(t)=b(t)$
    \end{center}
    }%
}
\medskip

\noindent Dove $x(t)$ rappresenta la grandezza incognita, $a_i$ rappresentano i coefficienti costanti e $b(t)$ è il termine noto(generatori ecc...).

\noindent La soluzione del nostro sistema sarà quindi formata dalla somma di due contributi:
\begin{enumerate}
    \item La soluzione particolare $p(t)$, che avrà la stessa evoluzione temporale di $b(t)$. (se $b(t)=k$ anche $p(t)=c$)
    \item La soluzione dell'omogenea associata $o(t)$, la quale rappresenta \textbf{l'evoluzione libera del sistema} con i generatori 
    passivati. Nella maggior parte dei casi 
    
    \noindent $o(t)=k_1e^{\lambda_1t}+k_2e^{\lambda_2t}+\dots+k_ne^{\lambda_nt}$ con $\lambda_1,\lambda_2,\dots,\lambda_n < 0$ per la convergenza
    per $t->\infty$. Per ogni esponenziale si calcola una costante di tempo $\tau_n=\dfrac{1}{\lambda_n}[s]$, il quale rappresenta la rapidità
    dell'evoluzione libera del sistema. Più piccolo sarà $\tau$, minore sarà il tempo di transitorio. Il transitorio per convenzione infatti,
    ha durata pari a $T=5max\{\tau_1,\tau_2,\dots,\tau_n\}$. 

    \noindent N.B. i $\tau_i$ non sono associati ai dispositivi dinamici, ma rappresentano un fenomeno transitorio di interazione tra i componenti 
    dinamici. (vedi circuito RLC)
\end{enumerate}

\pagebreak

\subsubsection{Circuito RC}

\begin{figure}[h!]
    \begin{center}
        \includegraphics[width=0.5\linewidth]{img/RC.jpg}
    \end{center}
\end{figure}


\end{document}
